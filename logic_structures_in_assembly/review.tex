\documentclass{beamer}
\usetheme{Goettingen}
\usepackage{algorithm, algorithmicx}
\title{Logic Structures in Assembly Language}
\author{Anonymous}
\institute{Peking University}
\date{2021}
\newtheorem{rmk}{Remark}
\newcommand{\SubItem}[1]{
    {\setlength\itemindent{15pt} \item[-] #1}
}
\begin{document}

    \frame{\titlepage}
    \begin{frame}{Outline}
        \tableofcontents
    \end{frame}

    \section{Commands}
    \begin{frame}{Basic Assembly Commands}
        In Lesson 5, we learnt a bunch of new features, syntax and commands. Now let's take a little time to go over them once again.
    \end{frame}
        \subsection{Program Counter}
        \begin{frame}{Program Counter}
            \begin{itemize}
                \item Definition: A register storing the address of the next command.
                \item<2-> Only an \alert<2->{abstract concept}; Has different implementations, like \%rip(RIP) in x86.
                \item<3-> Cannot be directly written (read-only).
            \end{itemize}
            \only<4->{
                \begin{rmk}
                Can be used to access data stored in memory with relative address difference.
                \end{rmk}
                \begin{example}
                movq 8(\%rip) \%rax 
                \end{example}
            }
        \end{frame}
        \subsection{Condition Codes}
        \begin{frame}{Condition Codes}
            \begin{itemize}
                \item Synonym: State Flag
                \item<2-> Four Condition Codes:

                    \begin{tabular}{|c|c|c|}
                        \hline
                        &Full Name&Meaning\\
                        \hline\hline
                        CF&Carry Flag&unsigned overflow\\
                        \hline
                        ZF&Zero Flag&zero\\
                        \hline
                        SF&Sign Flag&negative result\\
                        \hline
                        OF&Overflow Flag&signed overflow\\
                        \hline
                    \end{tabular}

                \item<3-> Programmers should know which sort of (signed, unsigned) data they are dealing with. No checker inside CPU.
            \end{itemize}
        \end{frame}
        \begin{frame}{How condition codes are set?}
            The state of condition code depends on the result of the last arithmetic command.
            \only<2->{
            \begin{example}
                Suppose all the flags are 0 at first, doing\\
                sub \%rax \%rax\\
                mov \%rbx \%rax\\
                only changes ZF to be 1.
            \end{example}
        }
        \only<3->{
            \begin{rmk}
                \textbf{leaq} doesn't change the condition code.
            \end{rmk}
        }
        \only<4->{
            \begin{rmk}
                \textbf{inc} and \textbf{dec} set OF and ZF flags, but leave \textbf{CF} untouched.
            \end{rmk}
        }
        \end{frame}
        \subsection{Compare}
        \begin{frame}{Compare}
            There are two commands designed explicitly for setting the condition code: cmp (compare) and test.
            \begin{block}{definition}
                cmpX S2 S1 ($X \in \{b, w, l, q\}$): Do S1 - S2 without storing the result.
            \end{block}
            \only<2->{
            \begin{rmk}
                Used to compare the order of two (un)signed numbers.
            \end{rmk}
        }
        \end{frame}
        \subsection{Test}
        \begin{frame}{Test}
            \begin{block}{definition}
                cmpX S2 S1 ($X \in \{b, w, l, q\}$): Do S1 \& S2 without storing the result.
            \end{block}
            \only<2->{
            \begin{rmk}
                Useful with a \textbf{mask} to check if some given digits are all 0.
            \end{rmk}
        }
            \only<3->{
            \begin{rmk}
                Frequently used as testq \%rax \%rax to check if a number if zero. (Shorter than cmpq \%rax 0!)
            \end{rmk}
        }
        \end{frame}
        \subsection{Conditions}
        \begin{frame}{Conditions}
            \begin{itemize}
                \item Canonical
                    \begin{itemize}
                        \item n: not        ~
                        \item e: equal      ZF 
                        \item z: zero       ZF
                    \end{itemize}
                \item<2-> Signed
                    \begin{itemize}
                        \item g: greater    ~(SF $\oplus$ OF)\&ZF
                        \item l: less       SF $\oplus$ OF
                    \end{itemize}
                \item<3-> Unsigned
                    \begin{itemize}
                        \item a: above      ~CF\&~ZF
                        \item b: below      CF
                    \end{itemize}
            \end{itemize}
        \only<4->{
        \begin{rmk}
            \textbf{n} is used as a prefix to negate the conditions that follow.   
        \end{rmk}
        \begin{rmk}
            \textbf{e} can be used as a suffix to a,b,g,l representing 'or equal'.
        \end{rmk}
    }
        \end{frame}
        \subsection{Set}
        \begin{frame}{Set}
            We need a way to retrieve and manipulate the condition codes.
            \begin{block}{definition}
                setX D (X is a condition): moving the corresponding combination of condition codes into the lower-order byte of D.
            \end{block}
            \begin{rmk}
                Set doesn't clear the high-order bytes. 
            \end{rmk}
        \end{frame}
        \subsection{Jumps}
        \begin{frame}{Jumps}
            To implement more complicated structures while minimizing the size of the assembly program, jumps are introduced.
            \begin{block}{definition}
                jmp X: jump to an memory address specified by X (can be a label)\\
                jmp *R: jump to the memory address specified in register R\\
                jmp *(R): jump to the memory address specified in the memory specified by register R
            \end{block}
            \only<2->{
            \begin{rmk}
                PC-relative encoding have two advantages:
                \begin{itemize}
                    \item more compactly encoded instruction
                    \item more portable code
                \end{itemize}
            \end{rmk}
        }
        \end{frame}
        \subsection{Jump Tables}
        \begin{frame}{Jump Tables}
            We can also use a constant table to refer to what's the jump destination.
            \begin{block}{definition}
                An anchor label and an alignment specification.
            \end{block}
            \only<2->{
            \begin{block}{structure}
                .section .rodata /* stands for read-only data */\\
                    .align 8 /* specifies the alignment used in the table */\\
                .L4: /* the anchor label */\\
                .quad .L8\\
	            .quad .L2\\
	            ...
            \end{block}
            \begin{block}{usage}
                jmp .L2(,\%rax,8): jump to the label specified by \%rax
            \end{block}
            }
            \only<3->{
            \begin{rmk}
                Jump tables help perform jumps automatically.
            \end{rmk}
            }
        \end{frame}

    \section{Conditional Commands}
        \subsection{Conditional Jumps}
        \begin{frame}{Conditional Jumps}
            To harness the power of condition codes, conditional jumps are introduced.
            \begin{block}{definition}
                jX D (X is a condition): jump to the address specified by D if X is satisfied.
            \end{block}
            \begin{rmk}
                Conditional jump are the reason why we could implement logic structure.
            \end{rmk}
        \end{frame}
        \begin{frame}{rep\&repz}
            In the assembly code provided in the CSAPP textbook and the bomb lab, sometimes \textbf{ret} is written as \textbf{rep ret} or \textbf{repz ret}.\\
            So, why are they there in the first place?
            \only<2->{
            \begin{block}{Answer}
                They are only used for \alert<2->{AMD} CPUs since these CPUs cannot handle the return address if ret is reached from a conditional jump. In another word, rep and repz are meaningless occupiers that could avoid this situation from happening.
            \end{block}
            }
        \end{frame}
        \subsection{Conditional Moves}
        \begin{frame}{Condition Moves}
            Before diving into details why we introduce conditional moves, we first review the definition of conditional moves(cmov).
            \begin{block}{definition}
                cmovSX R1 R2 (S is the size and X is a condition): if X is satisfied, move data from R1 to R2.\\
                R1, R2 cannot both be memory address.
            \end{block}
            \begin{example}
                movq \%rdi, \%rax\\
                subq \%rsi, \%rax\\
                movq \%rsi, \%rdx\\
                subq \%rdi, \%rdx\\ 
                cmpq \%rsi, \%rdi\\
                cmovge \%rdx, \%rax\\
                ret
            \end{example}
        \end{frame}
        \begin{frame}{Condition Moves}
            So, why use cmov?
            \begin{itemize}
                \item<2-> Advantages\\
                    \begin{itemize}
                        \item Save space
                        \item Enhanced performance on pipelined CPUs (while jmp perform terribly) due to \textbf{condition code transfers}.
                    \end{itemize}
                \item<3-> Disadvantages\\
                    \begin{itemize}
                        \item May break the structure of assembly code
                        \item Increase the calculation workload
                        \item Forbidden in some cases
                    \end{itemize}
                \item<4-> Bad cases\\
                    \begin{itemize}
                        \item Expensive evaluations: function-involved
                        \item Risky computation: p ? *p : 0
                        \item Computations with side effects: +=, *=
                    \end{itemize}
            \end{itemize}
            \begin{rmk}<2->
                Pipelined CPU achieves high performance by overlapping the steps of the successive instructions.
            \end{rmk}
        \end{frame}

    \section{Logic Structures}
        \subsection{Conditional Branches (if)}
            \begin{frame}{if}
                \begin{block}{Implementation}
                    cmov, conditional jumps
                \end{block}
                \begin{example}
                    Implement a function that return abs of the first argument.
                \end{example}
                \begin{block}{Answer}
                    mov \%rdi \%rax
                    test \%rax \%rax\\
                    jge .L1\\
                    sub \$0 \%rax\\
                    .L1:\\
                      repz ret
                \end{block}
            \end{frame}
        \subsection{Loops (for, while)}
            \begin{frame}{for, while}
                \begin{block}{Implementation}
                    conditional jumps
                \end{block}
                \begin{example}
                    Implement a loop using \%rdi and \%rsi where the outer i iterates from 0 to 9 and the inner one iterates from i to 9 
                \end{example}
            \end{frame}
            \begin{frame}{for,while}
                \begin{block}{Answer}
                    mov \$0 \%rdi\\
                    jmp .CHECK1\\
                    .LOOP1:
                    mov \%rdi \%rsi\\
                    jmp .CHECK2\\
                    .LOOP2:\\
                    ... //do stuff\\
                    .CHECK2:\\
                    cmp \%rsi \$9\\
                    jle .LOOP2\\
                    .CHECK1:\\
                    cmp \%rdi \$9\\
                    jle .LOOP1
                \end{block}
            \end{frame}
        \subsection{Switch Statements (switch)}
            \begin{frame}{switch}
                \begin{block}{Implementation}
                    jump tables
                \end{block}
                \begin{example}
                    Implement a function on a 64x machine to jump to .L1 when \%rax is prime and .L1 when it is not. ($\%rax \in \{1,2,3,4\}$) 
                \end{example}
                \begin{block}{Answer}
                    .section .rodata\\
                    .align 8\\
                    .ANCHOR:\\
                    .quad .L0\\
                    .quad .L1\\
                    .quad .L1\\
                    .quad .L0\\
                    ...\\
                    jmp .ANCHOR(,\%rax,\$8)
                \end{block}
            \end{frame}
    \section{Exercises}
        \begin{frame}{Exercises}
            
        \end{frame}

\end{document}
